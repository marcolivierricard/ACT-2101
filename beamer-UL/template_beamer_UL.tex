\documentclass[aspectratio=169, 12pt, french]{beamer}
\usetheme{AnnArbor}
\usecolortheme{beaver}


% Utilisation de Tikz dans la présention
\usepackage{pgf}
\usepackage{tikz}
\usetikzlibrary{arrows,automata, shapes,chains, positioning, backgrounds}
\tikzstyle{line}=[draw] % here
% Définition d'une flèche
\tikzstyle{arrow} = [thick,->,>=stealth]

% test
\usepackage[comma,authoryear]{natbib}

% tabular environment
\usepackage{tabu} 

% Compatibilité avec Sweave
\input{src/sweave_config}



% make sure pause does not increase page count
\setbeamertemplate{footline}[frame number]{}

\setbeamertemplate{frametitle continuation}[from second][(suite)]

\setbeamertemplate{caption}[numbered]


% Encoding packages
\usepackage[utf8]{inputenc}
\usepackage[T1]{fontenc}
\usepackage{babel}	% Combine with \documentclass[12pt, french]{•}
\usepackage{lmodern}


% Packages mathématiques / newcommand
\usepackage{amsmath,amsthm,amssymb,latexsym,amsfonts}
\usepackage{empheq}
\usepackage{icomma}
\newcommand{\trainset}{\mathcal{D}_{\mathcal{T}}}
\newcommand{\testset}{\mathcal{D}_{\mathcal{V}}}

% Informations sur la présentation Beamer (Auteur, date, titre, etc.)
\title[\hyperlink{start}{ACT-2101}]{\Large Modélisation statistique d'évènements extrêmes}
\subtitle{\normalsize Application multivariée dynamique aux catastrophes naturelles de l'Océanie}
\author{\large \textbf{Marc-Olivier Ricard}}
\institute{\normalsize Sous la supervision de \\ Marie-Pier Côté}
\logo{\includegraphics[width=0.15\textwidth]{src/logo_UL.png}} 
\date{\normalsize 4 décembre 2019}


% Options

% Remove navigation symbol
\setbeamertemplate{navigation symbols}{}

%% Style de la bibliographie.
\bibliographystyle{francais}

\frenchbsetup{%
    StandardItemizeEnv=true,       
    ThinSpaceInFrenchNumbers=true, 
    og=«, fg=»                     
  }
 

\begin{document}


\begin{frame}
\titlepage
\end{frame}


\begin{frame}{Plan de la présentation}
\tableofcontents
\end{frame}

\section{Mise en contexte}



\begingroup
\large
\begin{frame}{Motivations}
\begin{itemize}
	\item Intérêt recherche
	\item Stage été 2019
	\item Maîtrise
	\item Problématiques de l'industrie
	\item Black Swans
\end{itemize}
\end{frame}
\endgroup


\section{Notions préliminaires}

\subsection{Théorie classique des valeurs extrêmes}
\begin{frame}{Théorie classique des valeurs extrêmes}
Soit $M_n = \max\{X_1, \dots, X_n\}$, le maximum des n observations indépendantes et de distribution commune. 
S'il existe des séquences de constantes $\{a_n > 0\}$ et $\{b_n\}$ tel que $$P\Bigg\{\frac{M_n - b_n}{a_n} \le z \Bigg\}\to G(z), \ \ \  \text{quand} \ \ {n \to \infty}$$ où $G$ est une fonction de répartition non-dégénératrice. Alors, $G$ appartient à une des distributions de valeur extrême, soit les lois Gumbel, Fréchet et Weibull. \\~\\
Donc, peu importe la distribution ${F_X}_i$, les trois dernières lois sont les seules distributions limites pour $M^*_n$.
\end{frame}

\begin{frame}
Une meilleure analyse est possible en combinant en une seule distribution les distributions Gumbel, Fréchet et Weibull:\\~\\
Famille de distributions d'extremum généralisée (GEV)
\begin{equation}
\begin{gathered}
G(z) = \exp \Bigg\{ - \Big[ 1 +\xi\Big(\frac{z-\mu}{\sigma}\Big) \Big]^{-1/\xi}  \Bigg\}
\end{gathered}
\end{equation} 
Comme on peut voir, le modèle possède trois paramètres: $\mu$ (paramètre de position), $\sigma$ (paramètre d'échelle), $\xi$ (paramètre de forme).
\end{frame}

\begin{frame}
Tout ceci mène à une première méthodologie pour modéliser les valeurs extrêmes:
\begin{itemize}
\item Les données sont groupées en séquence de longueur $n$
\item Le maximum de chaque séquence est calculé
\item Une distribution \textit{GEV} est calibrée à ces maximums
\item La distribution peut être manipulée pour obtenir différentes statistiques
\end{itemize}
\end{frame}


\subsection{Méthode par l'approche POT}
\begin{frame}{Méthode par l'approche POT}
Soit $X_1,\dots, X_n$, une séquence de variables indépendantes, identiquement distribuées et avec fonction de répartition marginale $F$. Nous considérons comme évènements extrêmes ceux qui dépassent un certain seuil $u$.\\~\\
Nous savons déjà que $\Pr\{ \max\{X_1,\dots,X_n\} \le z \}\approx G(z)$
où
\begin{equation*}
{G(z) = \exp \Bigg\{ - \Big[ 1 +\xi\Big(\frac{z-\mu}{\sigma}\Big) \Big]^{-1/\xi}  \Bigg\}}
\end{equation*}
\end{frame}

\begin{frame}
Ensuite, pour $u$ assez grand, la fonction de répartition de $X-u\ |\ X>u$ est environ:
\begin{equation}
H(y) = 1 - \Big(1+\frac{\xi y}{\tilde\sigma}\Big)^{-1/\xi}
\end{equation}
où
\begin{equation*}
{\tilde{\sigma} = \sigma + \xi(u-\mu)}
\end{equation*}
Il s'agit ici de la famille de distributions Pareto généralisée.\\~\\
Les paramètres de $H$ sont uniquement déterminés par ceux de la distribution \textit{GEV}. $\xi$ conserve la même valeur, par exemple.
\end{frame}

\begin{frame}
Tout ceci mène à une deuxième méthodologie pour modéliser les valeurs extrêmes:
\begin{itemize}
\item Les données brutes représentent une séquence de valeurs iid  $x_1,\dots,x_n$
\item Les valeurs extrêmes sont identifiées en définissant un seuil de grande valeur $u$
\item Les observations qui dépassent $u$ sont définies par $x_{(1)},\dots,x_{(k)}$ et les excès de seuil par $y_j = x_{(j)} - u,\ j=1,\dots,k$
\item Les $y_j$ sont reconnues comme des réalisations indépendantes d'une variable aléatoire dont la distribution peut être approximée par un membre de la famille Pareto généralisée
\end{itemize}
\end{frame}

\begin{frame}
Un des défis de cette procédure est le choix de la valeur du seuil, où il faut trouver une bonne balance entre la variance et le biais du modèle. \\~\\
Deux méthodes:
\begin{itemize}
\item \textit{Mean residual plot}
\item Estimer les paramètres du modèle pour différents seuils
\end{itemize}
\end{frame}


\begin{frame}
Niveaux de retour:
\begin{equation}\
\begin{split}
{\Pr\{ X>x\ |\ X>u\} = \Big[ 1 + \xi \Big( \frac{x-u}{\sigma}\Big)\Big]^{-1/\xi}}\\
{\Rightarrow \Pr\{X>x\} = \zeta_u \Big[1 + \xi \Big( \frac{x-u}{\sigma}\Big)\Big]^{-1/\xi}}
\end{split}
\end{equation}
où $\zeta_u = \Pr\{X>u\}$. Ensuite, le niveau $x_m$ qui est dépassé en moyenne une fois chaque $m$ observations est obtenu comme suit:
\begin{equation}
\begin{split}
{\frac{1}{m} = \zeta_u \Big[1 + \xi \Big( \frac{x-u}{\sigma}\Big)\Big]^{-1/\xi}}\\
{\Rightarrow x_m = u + \frac{\sigma}{\xi} \Big[(m\zeta_u)^\xi -1\Big],\ \ x_m>u}
\end{split}
\end{equation}
\end{frame}

\section{Article étudié}

\begin{frame}{Article étudié}
Article choisi: \cite{chavez2016extreme}
\begin{itemize}
\item Tout de même récent
\item Longueur et difficulté convenables
\item Sujet spécifique intéressant
\item Implémentation en \textsf{R}
\end{itemize}
\end{frame}

\subsection{Introduction}
\begin{frame}{Introduction}
L'article choisi présente une méthodologie dans laquelle la modélisation des valeurs extrêmes dépend de variables exogènes associées à la variable réponse qui est modélisée.
\begin{itemize}
\item Ce sont les paramètres du modèle de valeurs extrêmes choisi qui dépendront des covariables disponibles.
\item Vise à améliorer la calibration des lois aux données et du même coup rendre l'inférence plus crédible et précise.
\end{itemize}
\end{frame}

\begin{frame}
\begin{itemize}
\item Processus de Poisson non-homogène 
\item Modèles additifs généralisés, splines de lissage et méthode du maximum de vraisemblance pénalisé
\item Pareto généralisée non-stationnaire dont les paramètre dépendent des variables exogènes
\item Algorithme basé sur les paramètres orthogonaux
\item  L'idée de dépendance entre les paramètres du modèle et des covariables a déjà élaborée dans \cite{coles2001introduction}, ce qui est réellement nouveau dans l'approche proposée est la dépendance semi-paramétrique avec les splines de lissage
\end{itemize}
\end{frame}

\begin{frame}
Critères:
\begin{itemize}
\item Les données représentent des pertes aléatoires encourus à des temps aléatoires.
\item Le but de l'analyse est d'estimer la distribution de la perte globale à travers le temps.
\item Les données contiennent un nombre suffisant d'observations.
\item Les données contiennent des évènement extrêmes.
\item Les données contiennent des covariables significatives.
\end{itemize}
Notes:
\begin{itemize}
\item Gains
\item $\beta$
\end{itemize}
\end{frame}
\subsection{Approche dynamique de la théorie des valeurs extrêmes}
\begin{frame}{Approche dynamique de la théorie des valeurs extrêmes}
La dépendance entre les paramètres et les covariables peut être paramétrique, non-paramétrique ou bien semi-paramétrique et elle peut également inclure des intéractions entre les variables. L'idée générale du modèle peut être représentée par l'équation suivante:
\begin{equation}
g_k(\theta_k) = f_k(x) + h_k(t), \ \ k \in \{1, \dots, p \}
\end{equation}
où:
\begin{itemize}
\item $\theta$ est le vecteur des $p$ paramètres du modèle.
\item $g_k$ est une fonction de lien.
\item $f_k$ est la fonction pour les différents niveaux d'une variable catégorique.
\item $h_k$ est soit une fonction linéaire paramétrique ou bien une fonction lisse non-paramétrique de $t$.
\end{itemize}
\end{frame}


\begin{frame}
On assume que le nombre d'excès du seuil $u$ sélectionné suit un processus de Poisson non-homogène avec comme taux:
 \begin{equation}
 \lambda = \lambda(x, t) = \exp(f_\lambda(x) + h_\lambda(t))
  \end{equation}
  
L'application d'un modèle additif généralisé mène à un estimé $\hat\lambda$.
\end{frame}

\begin{frame}
Il faut s'assurer que les procédures de calibration des paramètres $\xi$ et $\beta$ aux données convergent. Pour ce faire, il faut absolument que ces deux paramètres soient orthogonaux par rapport à l'information de Fisher.
\begin{equation}
 \begin{split}
  \nu = \log((1+\xi)\beta), & \ \xi > -1 \\ 
  \Rightarrow \beta = \frac{\exp(\nu)}{1+\xi}
  \end{split}
  \end{equation}
\end{frame}

\begin{frame}
Dans la même ordre d'idée que pour $\lambda$, on définit $\xi$ et $\nu$ comme suit:
\begin{equation}\label{eq:8}
 \xi = \xi(x, t) = f_\xi(x) + h_\xi(t)
\end{equation}
  
 \begin{equation}\label{eq:9}
 \nu = \nu(x, t) = f_\nu(x) + h_\nu(t)
 \end{equation}
 
  \begin{equation}
  \Rightarrow \beta = \beta(x, t) =  \frac{\exp(\nu(x,t))}{1+\xi(x,t)}
 \end{equation}
\end{frame}

\begin{frame}
Les équations \ref{eq:8} et \ref{eq:9} sont donc celles qui seront estimées avec les excès de seuil disponibles pour être en mesure d'obtenir $\hat\xi$ et $\hat\beta$. Ces estimateurs sont donc des estimateurs basés sur les estimateurs $\hat{f}_\xi, \hat{h}_\xi, \hat{f}_\nu$ et $ \hat{h}_\nu$. Ces derniers estimateurs sont obtenus à partir des vecteurs observés $z_i = (t_i, x_i, {y_t}_{i})$ où:
\begin{itemize}
\item $i \in \{1, \dots, n\}$.
\item $0 \leq t_1 \le \cdots \le t_n \leq T$ représente les temps d'excès de seuil.
\item $x_i$ représente le vecteur des covariables.
\item ${y_t}_{i}$ représente les réalisations de la variable aléatoire ${Y_t}_{i}$.
\item ${Y_t}_{i}$ représente les excès de seuil $u$.
\end{itemize}
\end{frame}

\begin{frame}
\begin{equation}
\widehat{\text{VaR}_\alpha} = u + \frac{\hat\beta}{\hat\xi} \Bigg(\Bigg( \frac{1-\alpha}{\hat\lambda/n'}\Bigg)^{-\hat\xi} -1 \Bigg)
\end{equation}

\begin{equation}
\widehat{\text{ES}_\alpha} =
\begin{cases}
\frac{\widehat{\text{VaR}_\alpha} + \hat\beta + u\hat\xi}{1-\hat\xi}, & \hat{\xi} \in (0,1) \\
\infty, & \hat\xi 	\ge 1
\end{cases}
\end{equation}
où $n' = n'(x,t)$ représente le nombre total de pertes pour les covariables $x$ et pour le temps $t$.\\~\\ Cependant, en pratique, il est plus utile de modéliser directement $\rho = \rho(x, t) = \lambda(x,t)/n'(x, t)$ qui représente le taux d'excès de seuil pour $x$ et $t$.
\end{frame}


%% Application à des données réelles
% le Sweaveopt


\section{Application à des données réelles}

\subsection{Analyse de données}



\begin{frame}{Analyse de données}
Pour mettre en application l'approche proposée par \cite{chavez2016extreme}, deux jeux de données du paquetage \texttt{CASdatasets} sont utilisés. Soit \texttt{auscathist} et \texttt{nzcathist}. \\~\\

Ces données représentent respectivement l'historique des catastrophes naturelles pour l'Australie ainsi que pour la Nouvelle-Zélande. Les prochaines diapositives présentent une analyse globale des données utilisées.
\end{frame}


\begin{frame}
\begin{figure}
\includegraphics[width=.8\textwidth]{images/fig-005.pdf}
\caption{Densité du logarithme du montant des catastrophes}
\end{figure}
\end{frame}

\begin{frame}
\begin{figure}
\includegraphics[width=.8\textwidth]{images/fig-006.pdf}
\caption{Nombre cumulatif de catastrophes par pays}
\end{figure}
\end{frame}

\begin{frame}
\begin{figure}
\includegraphics[width=.8\textwidth]{images/fig-007.pdf}
\caption{Évolution des catastrophes par pays}
\end{figure}
\end{frame}

\begin{frame}
% latex table generated in R 3.6.1 by xtable 1.8-4 package
% Mon Nov 25 16:50:46 2019
\begin{table}[ht]
\centering
\begin{tabular}{cccccccc}
  \hline
Type & N & Moyenne & Écart & Minimum & Médiane & Q3 & Maximum \\ 
  \hline
Bushfire & 26 & 143 & 259 & 7 & 39 & 134 & 1217 \\ 
  Cyclone & 37 & 325 & 675 & 2 & 88 & 217 & 3343 \\ 
  Earthquake & 9 & 56 & 91 & 0 & 25 & 47 & 286 \\ 
  Flood & 83 & 62 & 230 & 0 & 5 & 39 & 2010 \\ 
  Flood, Storm & 36 & 54 & 79 & 1 & 24 & 71 & 414 \\ 
  Hailstorm & 41 & 274 & 606 & 0 & 75 & 235 & 3511 \\ 
  Other & 12 & 121 & 296 & 2 & 6 & 50 & 1035 \\ 
  Power outage & 2 & 6 & 6 & 1 & 6 & 8 & 11 \\ 
  Storm & 86 & 97 & 228 & 0 & 28 & 57 & 1424 \\ 
  Tornado & 12 & 7 & 13 & 0 & 3 & 7 & 47 \\ 
  Weather & 4 & 11 & 11 & 2 & 6 & 13 & 27 \\ 
   \hline
\end{tabular}
\caption{Résumé statistique des montants (M USD) de catastrophes par Type} 
\label{tab:3.6}
\end{table}\end{frame}

\subsection{Approche classique}

\begin{frame}
\begin{itemize}
\item Par souci de manque de temps, les résultats obtenus avec l'approche classique ne seront pas montrés de façon détaillée. On conlut que cette approche n'est pas tout à fait adéquate dans le cas présent. 
\item Cette approche reste viable, rapide et peut être une bonne solution lorsque seulement les montants sont disponibles.
\item Un seuil de 10 M USD fut sélectionné
\end{itemize}
\end{frame}



\subsection{Approche dynamique à deux variables}

\subsection{Approche dynamique à trois variables}



\section{Conclusion}

\section*{Bibliographie}
\nocite{*}

\begin{frame}[allowframebreaks]{Bibliographie}
\bibliography{biblio}
\end{frame}


\end{document}
